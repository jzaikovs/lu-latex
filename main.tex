
\documentclass{LU}

\title{Darba nosaukums}	   
\thesistype{Kursa / Bakalaura/ Maģistra darbs}
\author{Vārds Uzvārds}
\studentid{Studenta apliecības numurs}
\supervisor{Grāds Vārds Uzvārds}
\university{Latvijas Universitāte}
\faculty{Fakultātes Nosaukums}
\location{Rīga}


% apzīmējumu saraksts
\makeglossaries

\newglossaryentry{example}
{
    name=Example,
    description={Descrition of example}
}

\begin{document}

\maketitle

\begin{abstract}
    Īss un kodolīgs darba apraksts!

    Atslēgas vārdi: suns, kaķis, pele.
\end{abstract}
 

\selectlanguage{english}
\begin{abstract}
    Same abstract just in English.

    Keywords: dog, cat, mouse.
\end{abstract}
\selectlanguage{latvian}

\pagenumbering{gobble} 

\tableofcontents

%------------------------------------------------APZĪMĒJUMI---------------------------------------------------------

\printglossary[type=main,title={Apzīmējumu saraksts},toctitle={Apzīmējumu saraksts}]

\pagenumbering{arabic} % sākam numurēt lapas no apzīmējumu saraksta (3. pielikums iekš LU 03.02.2012, 1/38 ) 

\chapter*{Ievads} % * nepieliks numuru pie nosaukuma
\addcontentsline{toc}{chapter}{Ievads}
\pagestyle{plain}
\input{ievads}

%------------------------------------------------DARBS--------------------------------------------------------------

\chapter{Teorija}
\section{Problēmas pamatnostādne}
Šajā nodaļa mēs apskatam pētāmās problēmas teorētisko pusi.
Veicam daudz $copy + paste + alter \vee trnslate$.
Galvenais ir salikt labi daudz atsauču \cite{kant2018recent}.
Patiesībā katru teikumu varam uztver kā BS un atsauces ir veids kā pateikt, ka tas nav \gls{example}.

\section{Saistība ar citiem pētījumiem}

Te vajag aprakstīt saistību?
Vai šāda nodaļa vispār ir vajadzīga?

\chapter{Praktiskā daļa}
\input{risinajums}

%----------------------------------------------SECINĀJUMI----------------------------------------------------------

\chapter*{Secinājumi}
\addcontentsline{toc}{chapter}{Secinājumi}
\input{secinajumi}

%---------------------------------------------LIETERATŪRA----------------------------------------------------------
\renewcommand{\bibname}{Izmantotā literatūra un avoti}
\bibliographystyle{abbrv} % nekur nav minēts kādam jābūt atsaucu noformējumam
\bibliography{main}
\addcontentsline{toc}{chapter}{Izmantotā literatūra un avoti}


%----------------------------------------------PIELIKUMS----------------------------------------------------------

\begin{appendices}
\chapter*{Pielikums}
\input{pielikums}
\end{appendices}

%---------------------------------------------REĢISTRĀCIJAS LAPA (TODO)-------------------------------------------

\chapter*{Reģistrācijas lapa}
Kursa darbs "Kaut kas, kaut kas" izstrādāts Latvijas
Universitātes Fakultātes nosaukums.
\vspace{1.5\baselineskip}

Ar savu parakstu apliecinu, ka darbs veikts patstāvīgi, izmantoti tikai tajā norādītie informācijas
avoti.

Autors: \makebox[1.5in]{\hrulefill} Vārds Uzvārds
\vspace{1.5\baselineskip}

Rekomendēju/nerekomendēju darbu aizstāvēšanai (nevajadzīgo izsvītrot)

Darba vadītājs: Grāds Vārds Uzvārds \makebox[1.5in]{\hrulefill} \makebox[.25in]{\hrulefill}.06.2019.
\vspace{1.5\baselineskip}

Darbs iesniegts Fakultātes nosaukums
\medskip

Dekāna pilnvarotā persona:
\vspace{1.5\baselineskip}

Darbs aizstāvēts kursa darbu komisijas sēdē

Komisija:


\end{document}
